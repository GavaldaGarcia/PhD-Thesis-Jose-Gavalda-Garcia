\chapter*{\textcolor{red}{Thesis aims}}
\addcontentsline{toc}{chapter}{Thesis aims}
\markboth{Thesis aims}{}

% \begin{enumerate}

%     \item Can we define a protein structure representation that better captures statistical behaviour when multiple conformations are present? 
%     \begin{enumerate}
%         \item To address this....
        
%     \end{enumerate}
%     \item Can we describe this statistical behaviour based on data that more accurately reflects protein behaviour in solution?
%     \item Can AlphaFold’s local metric for prediction confidence (pLDDT) effectively describe ambiguous protein behaviour?
%     \item Do AlphaFold models reflect information obtained from protein NMR ensembles, experimental NMR metrics of dynamics, and interpreted molecular dynamics data?
%     \item Can we identify biases and limitations within AlphaFold models?

% \end{enumerate}

\begin{enumerate}
    \item \textcolor{red}{Coordinate-based representations of proteins are limited when describing proteins that adopt multiple conformations. Even when numerical methods (\textit{e.g.} Root Mean Squared Fluctuations) are employed to describe the flexibility of these regions, they cannot identify conformational change events. To address this:} 
    \begin{enumerate}
        \item \textcolor{red}{Can we develop a protein representation that better captures statistical behaviour across multiple conformational states?}
        \item \textcolor{red}{Can this statistical behaviour be described using data that more accurately reflects protein behaviour in solution?}
        \item \textcolor{red}{Can this representation identify residue-level changes in conformation related to changes in such behaviour?}
    \end{enumerate}
    
    \item \textcolor{red}{AlphaFold can produce highly-accurate protein structural models from amino-acid sequence, enabling the structural prediction across whole proteomes. Considering that this method was trained and benchmarked with protein structural models produced under cryogenic conditions: }
    \begin{enumerate}
        \item \textcolor{red}{Does AlphaFold's local confidence metric, pLDDT, effectively capture ambiguous protein behaviour?}
        \item \textcolor{red}{Can we identify in AlphaFold models a reflection of protein flexibility and dynamics, measured from protein NMR ensembles, experimental NMR metrics of dynamics, and interpreted molecular dynamics data?}
        \item \textcolor{red}{Can biases and limitations within AlphaFold models be identified?}
    \end{enumerate}

    \item \textcolor{red}{Our research group, bio2Byte, has developed multiple tools to study different aspects of protein dynamics. While these tools provide complementary predictions of protein dynamics, unified execution was not possible at the start of this thesis due to incompatible requirements and syntax. Considering these limitations: }
    \begin{enumerate}
        \item \textcolor{red}{Can we harmonise these tools into a unified suite to enable seamless execution?}
        \item \textcolor{red}{Can we improve the tools' deployment to ensure long-term usability and expand accessibility for users without programming proficiency?}
    \end{enumerate}
\end{enumerate}

% Main point is the problem/question. Subpoints are how we're fixing it


