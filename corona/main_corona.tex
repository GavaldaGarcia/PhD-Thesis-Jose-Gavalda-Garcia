\chapter{Online biophysical predictions for SARS-CoV-2 proteins.}\label{chapter:corona}
\chaptermark{Biophysical predictions for SARS-CoV-2}

Luciano Kagami $^{1}$, Joel Roca-Martinez $^{1,2,3}$, Jose Gavalda-Garcia $^{1,2}$, Pathmanaban Ramasamy $^{1,2,3,4,5}$, K. Anton Feenstra $^{6,7}$, and Wim Vranken $^{1,2,3}$
\\
\\
$^{1}$ Interuniversity Institute of Bioinformatics in Brussels, ULB-VUB, \\Brussels, Belgium.
\\
$^{2}$ Structural Biology Brussels, Vrije Universiteit Brussel, Brussels, Belgium.
\\
$^{3}$ VIB Structural Biology Research Centre, Pleinlaan 2, 1050 Brussels, Belgium.
\\
$^{4}$ VIB-UGent Center for Medical Biotechnology, VIB, 9000 Ghent, Belgium.
\\
$^{5}$ Department of Biomolecular Medicine, Faculty of Health Sciences and Medicine, Ghent University, 9000 Ghent, Belgium.
\\
$^{6}$ IBIVU – Center for Integrative Bioinformatics, Department of Computer Science, Vrije Universiteit Amsterdam, Amsterdam, 1081HV The Netherlands.
\\
$^{7}$ AIMMS – Amsterdam Institute for Molecules Medicines and Systems, Vrije Universiteit Amsterdam, Amsterdam, 1081HV The Netherlands.
\\
\\
DOI: 10.1186/s12860-021-00362-w
\vspace{1em}
\hrule
\vspace{1em}

\begin{abstract}
    \textbf{Background:} The SARS-CoV-2 virus, the causative agent of COVID-19, consists of an assembly of proteins that determine its infectious and immunological behavior, as well as its response to therapeutics. Major structural biology efforts on these proteins have already provided essential insights into the mode of action of the virus, as well as avenues for structure-based drug design. However, not all of the SARS-CoV-2 proteins, or regions thereof, have a well-defined three-dimensional structure, and as such might exhibit ambiguous, dynamic behaviour that is not evident from static structure representations, nor from molecular dynamics simulations using these structures.

    \textbf{Main:} We present a website (\burl{https://bio2byte.be/sars2/}) that provides protein sequence-based predictions of the backbone and side-chain dynamics and conformational propensities of these proteins, as well as derived early folding, disorder, $\beta$-sheet aggregation, protein-protein interaction and epitope propensities. These predictions attempt to capture the inherent biophysical propensities encoded in the sequence, rather than context-dependent behaviour such as the final folded state. In addition, we provide the biophysical variation that is observed in homologous proteins, which gives an indication of the limits of their functionally relevant biophysical behaviour.

    \textbf{Conclusion:} The \burl{https://bio2byte.be/sars2/} website provides a range of protein sequence-based predictions for 27 SARS-CoV-2 proteins, enabling researchers to form hypotheses about their possible functional modes of action.
\end{abstract}