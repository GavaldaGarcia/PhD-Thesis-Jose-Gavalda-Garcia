\chapter*{Abstract}

Proteins are molecules vital for the function of cells, where they perform a wide variety of tasks. They are composed of an amino acid chain that typically arranges into a stable three-dimensional fold, often represented as a single thermodynamically favourable conformation. However, proteins in physiological conditions are in fact better represented as an ensemble of conformations that reflects a more complex thermodynamic landscape. Protein function does indeed require a certain degree of dynamics. Experimental and computational techniques such as Nuclear Magnetic Resonance (NMR) spectroscopy and Molecular Dynamics (MD) simulations are employed to obtain such protein conformational ensembles and investigate the range and timescale of protein backbone motions. Despite these advancements, characterising and interpreting protein dynamics remains challenging.

This thesis presents a novel method to describe protein local conformation and backbone dynamics based on probabilistically-defined conformational states, acknowledging that an amino acid can coexist in multiple states. To this end, six low-energy conformational states were defined using solution NMR data from 1,322 proteins. Secondary structure assignments and interpreted chemical shifts were employed to curate six sets of backbone dihedral angles corresponding to each conformational state. A probability density function was fitted to these sets and used to analyse dihedral angles from MD simulations on 113 proteins. Any backbone dihedral pair can now be represented as a 6-dimensional vector of conformational state propensities. The mean difference between the vectors for each structure in an ensemble and their average vector is proportional to its ability to exist in multiple conformational states. This metric provides a dynamic interpretation of protein conformation and backbone dynamics, identifying regions with insufficient sampling and latent conformational propensities.

In the context of recent advances in protein structure prediction, particularly through DeepMind’s AlphaFold, this thesis also examines the capability of predicted structures to capture backbone dynamics. An initial comparison of AlphaFold's local accuracy metric, pLDDT, with predicted dynamics metrics from sequence data using the b2BTools suite revealed an inverse correlation with various dynamics metrics. A large-scale analysis with experimentally obtained dynamics metrics further validated this trend. The study identified cases where AlphaFold’s training data pre-processing resulted in the rigid prediction of flexible loops and highlighted that, while pLDDT distinguishes between rigid and dynamic regions, it fails to accurately represent subtle gradations in dynamics.

Finally, in the spirit of open science these tools were made available to the research community, so offering new perspectives of quantifying the dynamic nature of proteins.
