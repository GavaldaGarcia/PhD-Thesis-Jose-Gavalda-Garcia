\chapter*{Abstract}

Proteins are molecules vital for the function of cells, where they perform a wide variety of tasks. They are composed of an amino acid chain that typically arranges into a stable three-dimensional fold, often represented as a single thermodynamically favourable conformation. However, proteins in physiological conditions are in fact better represented as an ensemble of conformations that reflects a more complex thermodynamic landscape. Protein function does indeed require a certain degree of dynamics. Experimental and computational techniques such as Nuclear Magnetic Resonance (NMR) spectroscopy and Molecular Dynamics (MD) simulations are employed to obtain such protein conformational ensembles and investigate the range and timescale of protein backbone motions. Despite these advancements, characterising and interpreting protein dynamics remains challenging.

\textcolor{red}{To advance the characterisation of protein dynamics, this thesis presents a novel probabilistic approach to} describe protein local conformation and backbone dynamics based on probabilistically-defined conformational states, acknowledging that an amino acid can coexist in multiple states. To this end, six low-energy conformational states were defined using solution NMR data. Secondary structure assignments and interpreted chemical shifts were employed to curate six sets of backbone dihedral angles corresponding to each conformational state. A probability density function was fitted to these sets and used to analyse dihedral angles from MD simulations. Any backbone dihedral pair can now be represented as a vector of conformational state propensities, and is accompanied by the definition of a \textcolor{red}{novel} metric of conformational state variability\textcolor{red}{, even for regions with no defined stable structure.}

This thesis also examines the capability of predicted structures to capture backbone dynamics. An initial comparison of AlphaFold's pLDDT with predicted dynamics metrics from sequence data using the b2BTools suite revealed an inverse correlation with various dynamics metrics. A large-scale analysis with experimentally obtained backbone dynamics metrics further validated this trend. The study identified cases where AlphaFold’s training data pre-processing resulted in the rigid prediction of flexible loops and highlighted that, while pLDDT distinguishes between rigid and dynamic regions, it fails to accurately represent subtle gradations in dynamics.

Finally, in the spirit of open science these tools \textcolor{red}{have been} made available to the research community, so offering new perspectives of quantifying the dynamic nature of proteins. \textcolor{red}{The work in this thesis offers a quantitative probabilistic representation of protein conformational space and variability, and it assesses the ability of AlphaFold models to capture backbone dynamics, suggesting inherent limitations in current predictive models in reflecting flexibility.}