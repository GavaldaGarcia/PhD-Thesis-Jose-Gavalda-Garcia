\chapter*{Abstract}

A protein's fold in physiological conditions is better represented as an ensemble of conformations that reflects a complex thermodynamic landscape. Experimental and computational techniques such as Nuclear Magnetic Resonance (NMR) spectroscopy and Molecular Dynamics (MD) simulations are employed to obtain such protein conformational ensembles and investigate the range and timescale of protein backbone motions. Despite these advancements, characterising and interpreting protein \textcolor{red}{dynamic conformations} remains challenging.

\textcolor{red}{To advance in this characterisation, this thesis presents a novel probabilistic approach to describing} protein local conformation and backbone dynamics based on probabilistically-defined conformational states, acknowledging that an amino acid can coexist in multiple states. To this end, six low-energy conformational states were defined using solution NMR data\textcolor{red}{, thus capturing in-solution protein behaviour. A probability density function was fitted for each conformational state employing the backbone dihedrals extracted from NMR ensembles, then validated with MD simulations.} Any backbone dihedral pair can now be represented as a vector of conformational state propensities, and is accompanied by the definition of a \textcolor{red}{novel} metric of conformational state variability\textcolor{red}{, even for regions with no defined stable structure.}

\textcolor{red}{The latest generation of protein structure predictors, notably AlphaFold, is capable of producing highly-accurate structures from their amino-acid sequence. Their predictions are trained and evaluated employing protein structures obtained in cryogenic conditions, without explicit information on their backbone dynamics and flexibility. Consequently,}
 this thesis also examines the capability of \textcolor{red}{such} predicted structures to capture backbone dynamics \textcolor{red}{and flexibility}. An initial comparison of AlphaFold's pLDDT with predicted dynamics metrics from sequence data using the b2BTools suite revealed an inverse correlation with various dynamics metrics. \textcolor{red}{Then,} a large-scale analysis with experimentally obtained backbone dynamics metrics further validated this trend. The study identified cases where AlphaFold’s training data pre-processing resulted in the rigid prediction of flexible loops and highlighted that, while pLDDT distinguishes between rigid and dynamic regions, it fails to accurately represent subtle gradations in dynamics.

Finally, in the spirit of open science these tools \textcolor{red}{have been} made available to the research community, so offering new perspectives of quantifying the dynamic nature of proteins. \textcolor{red}{The work in this thesis offers a quantitative probabilistic representation of protein conformational space and variability, and it assesses the ability of AlphaFold models to capture backbone dynamics, suggesting inherent limitations in current predictive models in reflecting flexibility.}