\chapter*{Samenvatting}

Eiwitten zijn moleculen van vitaal belang voor de functie van cellen, waar ze een breed scala aan taken uitvoeren. Ze bestaan uit een aminozuurketen die zich typisch organiseert in een stabiele driedimensionale vouwing, vaak weergegeven als een enkele thermodynamisch gunstige conformatie. Eiwitten in fysiologische omstandigheden bestaan echter als een ensemble van conformaties die een complexer thermodynamisch landschap weerspiegelen. Eiwitten moeten inderdaad bewegen om te functioneren. Experimentele en computationele technieken zoals Nuclear Magnetic Resonance (NMR) spectroscopie en Molecular Dynamics (MD) simulaties worden gebruikt om eiwitconformatie-ensembles te verkrijgen en het bereik en de tijdschaal van eiwitten te onderzoeken. Ondanks deze technieken blijft het karakteriseren en interpreteren van de dynamiek van eiwitten een uitdaging.

Dit proefschrift presenteert een nieuwe methode om de lokale conformatie en dynamiek van eiwitten te beschrijven op basis van probabilistisch gedefinieerde conformatietoestanden, waarbij een aminozuur meerdere toestanden kan adopteren. Om dit doel te bereiken werden zes conformationele toestanden met lage energie gedefinieerd met behulp van NMR-gegevens van 1,322 eiwitten. De secundaire structuren en geïnterpreteerde NMR chemische verschuivingen werden gebruikt om zes sets van backbone-dihedrale hoeken te bepalen die overeenkomen met elke conformationele toestand. Een dichtheidsfunctie van de waarschijnlijkheid van deze toestanden werd ontwikkeld op basis van deze sets en gebruikt om dihedrale hoeken te analyseren van MD-simulaties op 113 eiwitten. Elk aminozuur kan zo worden weergegeven als een 6-dimensionale vector van hun conformationele toestanden. Het gemiddelde verschil tussen de vectoren voor elke structuur in een ensemble en hun gemiddelde vector is evenredig aan het vermogen om in meerdere conformationele toestanden te bestaan. Deze metriek biedt een nieuwe interpretatie van de conformatie van eiwitten en de dynamiek van aminozuren, waarbij regio's met onvoldoende bemonstering en latente conformationele neigingen kunnen worden geïdentificeerd.

In de context van recente ontwikkelingen in de voorspelling van eiwitstructuren, met name via DeepMind's AlphaFold, onderzoekt dit proefschrift ook het vermogen van dergelijke voorspelde structuren om de dynamiek van aminozuren te bepalen. Een vergelijking van AlphaFold's lokale nauwkeurigheidsmetriek, pLDDT, met voorspelde dynamische parameters onthulde een omgekeerde correlatie, die bevestigd werd door een grootschalige analyse met experimenteel verkregen parameters. De studie identificeerde gevallen waarin AlphaFold flexibele lussen als rigide voorspelde en onderstreept dat de pLDDT subtiele gradaties in dynamiek niet nauwkeurig kan weergeven.

Tot slot werden deze tools in de geest van ``open science'' beschikbaar gesteld aan de onderzoeksgemeenschap. Ze bieden zo nieuwe perspectieven voor het kwantificeren van de dynamische aard van eiwitten.


