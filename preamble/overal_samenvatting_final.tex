\chapter*{Samenvatting}

Eiwitten in fysiologische omstandigheden zijn dynamisch en worden daarom beter weergegeven als een ensemble van conformaties die hun complexe thermodynamische landschap weerspiegelen. Experimentele en computationele technieken zoals Nuclear Magnetic Resonance (NMR) spectroscopie en Molecular Dynamics (MD) simulaties verschaffen informatie over dergelijke eiwitconformatie-ensembles en het bereik en de tijdschaal van eiwitbewegingen. Het karakteriseren en interpreteren van eiwitdynamiek en conformaties blijft echter een uitdaging.

Dit proefschrift presenteert een nieuwe benadering om de lokale conformatie en dynamiek van eiwitten te beschrijven op basis van probabilistisch gedefinieerde conformationele toestanden, waarbij rekening gehouden wordt met de meerdere conformaties die aminozuren kunnen aannemen. Zes conformationele toestanden met lage energie werden gedefinieerd op basis van NMR-gegevens, waarmee het gedrag van eiwitten in oplossing werd vastgelegd. Voor elke conformationele toestand werd een waarschijnlijkheidsdichtheidsfunctie gedefinieerd door backbone-dihedralen uit NMR-ensembles te gebruiken. Een set backbone-dihedralen voor een bepaald residu, uit grote ensembles of MD-gegevens, kan vervolgens worden weergegeven als een vector die neigingen voor conformationele toestanden beschrijft. Bovendien werd een nieuwe metriek van conformationele toestandsvariabiliteit gedefiniëerd die ook kan worden toegepast op regio's zonder stabiele structuur.

De AlphaFold-eiwitstructuurvoorspeller, ondanks zijn capaciteit om zeer nauwkeurige structuren te produceren uit de aminozuursequentie van eiwitten, mist informatie over de backbone-dynamiek en flexibiliteit van eiwitten. Dit proefschrift onderzoekt ook het vermogen van dergelijke voorspelde structuren om backbone-dynamiek en flexibiliteit vast te leggen. De pLDDT-betrouwbaarheidsmetriek van AlphaFold correleert omgekeerd evenredig met dynamische metrieken die zijn voorspeld uit sequentiegegevens. Een verdere grootschalige analyse met experimenteel verkregen backbone-dynamiekinformatie bevestigt deze trend. De studie identificeerde gevallen waarin AlphaFold's trainingsdata pre-processing resulteerde in de rigide voorspelling van flexibele lussen. Onze resultaten benadrukken dat, hoewel de pLDDT onderscheid maakt tussen rigide en dynamische regio's, het er niet in slaagt om subtiele gradaties in dynamiek nauwkeurig weer te geven.

Het werk in dit proefschrift biedt daarom een kwantitatieve probabilistische weergave van de conformationele ruimte en variabiliteit van eiwitten, en beoordeelt het vermogen van AlphaFold-modellen om backbone-dynamiek vast te leggen, wat suggereert dat er inherente beperkingen zijn in de huidige voorspellende modellen in relatie tot flexibiliteit. In de geest van open science zijn alle tools en resultaten van deze studie gemakkelijk toegankelijk gemaakt voor de onderzoeksgemeenschap, waardoor nieuwe perspectieven worden geboden voor het kwantificeren van de dynamische aard van eiwitten.