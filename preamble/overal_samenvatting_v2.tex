\chapter*{Samenvatting}

Eiwitten zijn moleculen die van vitaal belang zijn voor de functie van cellen, waar ze een breed scala aan taken uitvoeren. Ze bestaan uit een aminozuurketen die zich typisch organiseert in een stabiele driedimensionale vouwing, vaak weergegeven als een enkele thermodynamisch gunstige conformatie. In fysiologische omstandigheden worden eiwitten echter beter weergegeven als een ensemble van conformaties, wat een complexer thermodynamisch landschap weerspiegelt. Voor hun functie is een zekere mate van dynamiek noodzakelijk. Experimentele en computationele technieken zoals Nuclear Magnetic Resonance (NMR) spectroscopie en Molecular Dynamics (MD) simulaties worden gebruikt om dergelijke eiwitconformatie-ensembles te verkrijgen en om het bereik en de tijdschaal van de bewegingen van de eiwitstructuur te onderzoeken. Ondanks deze vooruitgangen blijft het karakteriseren en interpreteren van eiwitdynamiek een uitdaging.

Om de karakterisering van eiwitdynamiek te bevorderen, presenteert dit proefschrift een nieuwe probabilistische benadering om lokale conformaties en backbone-dynamiek van eiwitten te beschrijven, gebaseerd op probabilistisch gedefinieerde conformationele toestanden, waarbij een aminozuur in meerdere toestanden kan coëxisteren. Hiervoor werden zes lage-energie conformationele toestanden gedefinieerd op basis van NMR-gegevens. Secundaire structuurtoewijzingen en geïnterpreteerde chemische verschuivingen werden gebruikt om zes sets van backbone-dihedrale hoeken te bepalen, die overeenkomen met elke conformationele toestand. Een waarschijnlijkheidsdichtheidsfunctie werd op deze sets toegepast en vervolgens gebruikt om dihedrale hoeken van MD-simulaties te analyseren. Elk paar backbone-dihedralen kan nu worden weergegeven als een vector van de conformationele toestandkansen, vergezeld van een nieuwe metriek voor conformationele toestandvariabiliteit, zelfs voor regio's zonder stabiele structuur.

Dit proefschrift onderzoekt ook de mogelijkheid van voorspelde structuren om backbone-dynamiek vast te leggen. Een eerste vergelijking van AlphaFold's pLDDT met voorspelde dynamische metriek uit sequentiegegevens, met behulp van de b2BTools suite, toonde een omgekeerde correlatie met verschillende dynamiekmetriek. Een grootschalige analyse met experimenteel verkregen backbone-dynamiekmetriek bevestigde deze trend. De studie identificeerde gevallen waarin AlphaFold's pre-processing van trainingsgegevens leidde tot het rigide voorspellen van flexibele lussen en benadrukte dat, hoewel pLDDT onderscheid maakt tussen rigide en dynamische regio's, het subtiele gradaties in dynamiek niet nauwkeurig kan weergeven.

In de geest van open science zijn deze tools beschikbaar gesteld aan de onderzoeksgemeenschap, waardoor nieuwe perspectieven ontstaan voor het kwantificeren van de dynamische aard van eiwitten. Dit proefschrift biedt een kwantitatieve probabilistische representatie van de conformationele ruimte en variabiliteit van eiwitten en beoordeelt het vermogen van AlphaFold-modellen om backbone-dynamiek vast te leggen, waarbij inherente beperkingen worden gesuggereerd van huidige voorspellende modellen in het weergeven van flexibiliteit.